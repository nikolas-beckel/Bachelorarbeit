%%%%%%%%%%%%%%%%%%%%%%%%%%%%%%%%%%%%%%%%%%%%%%%%%%%%%%%%%%%%%%%%%%%%%%%%%%%%%%%
%% LaTeX-Vorlage für Abschlussarbeiten                                       %%
%% (TH Köln -Campus Gummersbach, Fak. 10)                                    %%
%%                                                                           %%
%% Gemäß dem Merkblatt zur Anfertigung von Projekt-, Bachelor-, Master- und  %%
%% Diplomarbeiten der Fakultät 10 von Frau Prof. Dr. Halfmann &              %%
%% Herr Prof. Dr. Rühmann (Version vom 27.01.2008)                           %%
%%                                                                           %%                                                                            
%% Bitte sprechen Sie unbedingt mit Ihrer Betreuerin bzw. Ihrem Betreuer     %%
%% bezüglich der Ausgestaltung Ihrer Arbeit!                                 %%
%%                                                                           %%
%%                                                                           %%
%% MERKKASTEN IN DIESER VORLAGE:                                             %%
%% In dieser Vorlage finden Sie Merkkasten, die Ihnen Informationen          %%
%% zu bestimmten, formalen Aspekten geben. Sprechen Sie immer auch mit       %% 
%% Ihrer Betreuerin bzw. Ihrem Betreuer dazu an.                             %%                       
%% Für die eigene Verwendung der Vorlage entfernen oder kommentieren Sie die %%
%% Merkkasten. Die betreffenden Bereiche für die Merkkasten in der Vorlage   %%
%% sind wie folgt kommentiert: <MERKKASTEN> ... </MERKKASTEN>.               %%                            %%                                                                           %%
%%                                                                           %%
%% LIZENZ:                                                                   %%
%% Diese Vorlage darf nicht kommerziell verbreitet                           %%
%% werden. Eine nicht-kommerzielle Weitergabe ist                            %% 
%% gestattet.                                                                %%
%%                                                                           %%
%% Von Ludger Schönfeld, M. Sc.,                                             %%
%% 2014-2017                                                                 %%
%%%%%%%%%%%%%%%%%%%%%%%%%%%%%%%%%%%%%%%%%%%%%%%%%%%%%%%%%%%%%%%%%%%%%%%%%%%%%%%

%%%%%%%%%%%%%%%%%%%%%%%%%%%%%%%%%%%%%%%%%%%%%
%% HEADER                                  %%
%%%%%%%%%%%%%%%%%%%%%%%%%%%%%%%%%%%%%%%%%%%%%
\documentclass[a4paper,12pt,oneside]{article}
% Optionen:
% - a4paper => DIN A4-Format
% - 12pt    => Schriftgröße (weitere  
%              grundlegende Fontgrößen: 10pt, 11pt)
% - oneside => Einseitiger Druck

%% Verwendete Pakete:
\usepackage[ngerman]{babel} % für die deutsche Sprache
\usepackage{caption} % Für schönere Bildunterschriften
\usepackage[T1]{fontenc} % Schriftkodierung (Für Sonderzeichen u.a.)
\usepackage[utf8]{inputenc} % Für die direkte Eingabe von Umlauten im Editor u.a.
\usepackage{fancyhdr} % Für Kopf- und Fußzeilen
\usepackage{lscape} % Für Querformat

%% Schriften (Beispiele)
%% Weitere LaTeX-Schriften im "LaTeX Font Catalogue"
%% unter: http://www.tug.dk/FontCatalogue/.
%% ACHTUNG: Ggf. müssen Schriften noch installiert 
%% werden!

% Serifen-Schriften:
\usepackage{lmodern} % Schriftart "Latin Modern"
%\usepackage{garamond} % Schriftart "Garamond"

%Sans Serif-Schriften:
%\usepackage[scaled]{uarial}
%\usepackage[scaled]{helvet}
%%--------------
\usepackage[normalem]{ulem} % Für das Unterstreichen von Text z.B. mit \uline{}
\usepackage[left=3cm,right=2cm,top=1.5cm,bottom=1cm,
textheight=245mm,textwidth=160mm,includeheadfoot,headsep=1cm,
footskip=1cm,headheight=14.599pt]{geometry} % Einrichtung der Seite 

\usepackage{graphicx} % Zum Laden von Graphiken
% INFO: Graphiken einbinden
%
% \includegraphics[scale=1.00]{dateiname}
%
% => Ausgabeformat: PDF-Dokument:
%    Es können die folgenden (Graphik-)formate eingebunden
%    werden: .jpg, .png, .pdf, .mps
% 
% => Ausgabeformat: DVI/PS:
%    Folgende (Graphik-)formate werden unterstützt:
%    .eps, .ps, .bmp, .pict, .pntg
\usepackage{epstopdf}

% Pakete für Tabellen
\usepackage{tabularx} % Einfache Tabellen
\usepackage{longtable} % Tabellen als Gleitobjekte (für die Aufteilung bei langen 
 %Tabellen über mehrere Seiten)
\usepackage{multirow} % Für das Verbinden von Zeilen innerhalb einer Tabelle mit
 % \multirow{anzahl}{*}{Text}

% (Zusatz-)Pakete für Formeln
\usepackage{amsmath}
\usepackage{amsthm}
\usepackage{amsfonts}

\usepackage{setspace} % Paket zum Setzen des Zeilenabstandes
% INFO: Zeilenabstand setzen:
%
% Befehle:
% - \singlespacing  => 1-zeilig (Standard)
% - \onehalfspacing => 1,5-zeilig
% - \doublespacing  => 2-zeilig 
\onehalfspacing % Zeilenabstand auf 1,5-zeilig setzen

% Farbboxen (für die Merkkästen in dieser Vorlage):
\usepackage{tcolorbox}
\tcbset{colback=white,colframe=orange,
        fonttitle=\bfseries}

\usepackage[colorlinks,pdfpagelabels,pdfstartview=FitH,
bookmarksopen=true,bookmarksnumbered=true,linkcolor=black,
plainpages=false,hypertexnames=false,citecolor=black]{hyperref} % Für Verlinkungen
% INFO: Verlinkungen mit dem hyperref-Paket:
%
% Die Angabe von URLs mit dem Befehl \url{} erlaubt einen
% gesonderten Umgang mit Weblinks. Denn die Links werden verlinkt.
% Auch erfolgt automatisch am Zeilenende ein Umbruch des Links.
% Es ist auch nicht erforderlich, Sonderzeichen in der URL manuell zu 
% entschärfen.
%
% TIPP: Sollte ein Umbuch bei einem Link nicht automatisch erfolgen, so kann
% das daran liegen, dass ein/mehrere Zeichen zusätzlich angegeben werden müssen,
% an dem der Link umbrochen werden kann.
% Dies kann mit folgendem Befehl erfolgen (Beispiel):
% \renewcommand*\UrlBreaks{\do-\do_}

% Das Paket "biblatex" für autom. 
% Literaturverzeichnisse:
\usepackage{csquotes} % Für sprachangepasste Anführungszeichen
\usepackage[backend=biber,style=authoryear,citestyle=apa]{biblatex}
\addbibresource{bib/literatur.bib}

\title{Entwicklung von Darstellungs- und Interaktionsmöglichkeiten in Virtual Reality für das Cranach Digital Archive}
\author{Nikolas Beckel (11103435)}
\date{15. September 2020}
% Use \maketitle for generated title

%%%%%%%%%%%%%%%%%%%%%%%%%%%%%%%%%%%%%%%%%%%%%
%% DOKUMENT                                %%
%%%%%%%%%%%%%%%%%%%%%%%%%%%%%%%%%%%%%%%%%%%%%
\begin{document}
  \tableofcontents
  \newpage
  \section{Einleitung}
    % \begin{itemize}
    %   \item Das Thema vorzustellen
    %   \item Das Ziel vorzustellen
    %   \item Den Leser neugierig machen
    %   \item Die Relevanz zu beschreiben
    % \end{itemize}
      
    Virtual Reality ist nicht mehr nur Zukunftsmusik oder ein Anwendungsbereich für Forscher
    großer Konzerne. Immer mehr rückt Virtual Reality in den Massenmarkt. Wo zu Beginn
    leistungsfähige Computer benötigt wurden, entstehen nun All-in-One VR-Brillen\footnote{Oculus Quest 2 - All-in-One VR-Brille | \url{https://www.oculus.com/quest-2/} (21.09.2020)}.
    Die Anwendungsbereiche von Virtual Reality sind vielfältig: Gaming, Unterhaltung durch VR-Filme oder
    das Treffen von Freunden in virtuellen Welten\footnote{Mozilla Hubs | \url{https://labs.mozilla.org/projects/hubs/} (21.09.2020)}.
    Die Pandemie durch das Virus SARS-CoV-2 hat uns auch gezeigt, dass immer mehr digitale Lösungen
    benötigt werden. Durch solche Krisen bekommen plötzlich Anwendungsbereiche, die nicht als
    notwendig betrachtet worden, völlig neue Relevanz.\\
    Durch Virtual Reality ist man nicht mehr an einen Ort gebunden und kann gewisse Aktionen
    statt in der realen Welt, in einer virtuellen Welt ausführen. Da man von der realen Welt
    losgelöst ist, entstehen auch völlig neue Möglichkeiten, diese aufzuführen.\\
    In dieser Bachelorarbeit wird das digitale Archiv der Cranachs\footnote{Cranach Digital Archive | \url{http://lucascranach.org/} (21.09.2020)}
    als Ressource genutzt, um neue Möglichkeiten der Darstellung für Kunst und Kultur zu erforschen.
    Dabei sollen aktuelle Entwicklungen im musealen Bereich miteinbezogen und deren Erkenntnisse
    berücksichtigt werden.\\
    Auf Basis der recherchierten Ergebnisse sollen ein oder mehrere Prototypen entwickelt werden,
    die zeigen, wie man Virtual Reality im Bereich Kunst und Kultur anwenden kann. Dabei sollen die
    aktuellsten Technologien miteinbezogen und abgewägt werden, mit welcher Technologie die
    besten Ergebnisse erzielt werden können.

  \section{Grundlage und Datenbasis}
    In diesem Kapitel werden auf die Grundlagen von Virtual Reality, auf Lucas Cranach 
    der Ältere und auf das Cranach Digital Archive eingegangen. Letzteres stellt 
    die Datenbasis für dieses Projekt dar, welche in der Entwicklung der Prototypen 
    eingesetzt wird. Dieses Grundwissen wird für das nächste Kapitel benötigt, welches 
    sich tiefgründiger mit Virtual Reality und der Kunstwerke von Lucas Cranach
    beschäftigt.

    \subsection{Das Cranach Digital Archive}
      Das Cranach Digital Archive ist eine Initative und ein visionäres Forschungsprojekt, 
      welches sich zum Ziel gesetzt hat, alle relevanten Informationen, Dokumente und 
      Werke der Cranachs in einer digitalen Datenbank der Forschung und Öffentlichkeit zur 
      Verfügung zu stellen.
      So konnten bisher über 1600 Gemälde und 14000 Abbildungen digitalisiert und auf der 
      Webseite \url{lucascranach.org} veröffentlich werden, welche frei über das
      Internet zugänglich ist.
      Bei den digitaliserten Aufnahmen der Werke handelt es sich nicht nur um hochauflösende
      Gemälde, sondern auch Röntgenaufnahmen, Infrarotreflektogramme und
      Archivalien. Dadurch lassen sich verschiedene
      Gemälde der Cranachs aus einer völlig neuen Perspektive betrachten, denn durch die
      Infrarotreflektogramme lassen sich zum Beispiel Unterzeichnungen
      eines Gemäldes erkennen, welches das bloße Auge niemals sehen könnte. [\cite{heydenreich2017lucas}]
    \subsubsection{Lucas Cranach der Ältere}
      Lucas Cranach d. Ä. war nicht nur ein erfolgreicher Künstler während der Renaissance,
      sondern auch ein guter Freund Martin Luthers und ermöglichten gemeinsam eine moderne 
      Auffassung der Kunst.
      Lucas Cranach d. Ä. wurde 1472 als Sohn eines Malers namens Hans Moller geboren
      und wurde auch von ihm in der Zeichenkunst unterrichtet.
      Obwohl er früh mit dem Malen begonnen hat, wird Lucas Cranach d. Ä. erst in Wien
      um 1502 als Künstler greifbar. Um 1504/05 rum 
      verlässt er Wien und nimmt den Beruf des Hofmalers in Wittenberg für Friedrichs III. 
      entgegen, von dem er auch sein zukünftig verwendetes Wappen verliehen bekommen 
      hat.
      Dieses Wappen, eine \glqq geflügelte, bekrönte und einen Ring im Maul tragende
      Schlange\grqq{} [\cite[15]{heydenreich2017lucas}], sollte fortan als Signet von
      Lucas Cranach und seiner Werkstatt werden.
      Nach dem Tod von Friedrich III., diente er seinem Sohn und Nachfolger
      Johann I., welcher das Amt jedoch nur sieben Jahre lang führen konnte und
      schließlich an seinen Sohn, Johann Friedrich I. abgegeben hat.
      Bis zu Lucas Cranach d. Ä. Tods stand er im Dienst dieser drei Kurfürsten.
      Lucas Cranach d. Ä. zeichnete sind nicht nur in seiner Qualität der Gemälde aus,
      sondern war auch dafür bekannt produktiv zu arbeiten.
      Im Gegensatz zu seiner Konkurrenz malte er nicht einfach nur seine Gemälde an einem
      einfach Ort, sondern baute sich bereits in Wittenberg eine Werkstatt auf, die es
      ihm erlaubte, seine Produktivität auf eine neue Ebene zu bringen.
      Auch die Kunstgattung Druckgrafik, die Martin Luther als Graben bezeichnete, erlaubte
      Lucas Cranach d. Ä. einzelne Gemälde mehrfach zu drucken.
      Er war aber nicht nur als Künstler erfolgreich, sondern belegte in Wittenburg
      zwischen 1519 und 1544/45 das Amt des Ratsherren, einige Jahre davon auch als
      Kämmerer oder Bürgmeister.
      Die letzten Jahre Lucas Cranach d. Ä. waren durch den Schmalkaldischen Krieg
      geprägt und beeinflusst, da Johann Friedrich I. mit seinen Truppen gegen den
      Kaiser kämpfte. Nach einer Niederlage und
      Verlusten seines Territoriums, foltge Lucas Ranach d. Ä. Johann Friedrich I.
      in die Gefangenschaft und verblieb dort von 1547 - 1552.
      Nach der Gefangenschaft kehrte Lucas Cranach d. Ä. nach Weimar zurück und starb
      am 16. Oktober 1553 im Haus seiner Tochter.
      Zusammenfassend war Lucas Cranach d. Ä. nicht nur ein einfacher Künstler, sondern
      war aufgrund seiner Qualität, Produktivität und humanistischem Denken,
      welches sich vor allem in den Gemälden seiner Wiener Zeit wiederspiegelt,
      seiner Zeit voraus und legte einen wichtigen Grundstein für die moderne
      Kunst. [\cite{heydenreich2017lucas}]
    \subsubsection{Martin Luther als Junker Jörg}
      Als Martin Luther 1511 nach Wittenburg zurückkehrte und 1512 die Professur für
      Bibelauslegung übernommen hatte, traf er zum ersten mal auf Lucas Cranach d. Ä.
      Die beiden pflegten nicht nur engen Kontakt zueinander, sondern ergänzten sich auch
      in ihrer Arbeit. Martin Luthsers Gedanken der Reformation konnten nicht nur in
      Schrift verbreitet werden, sondern auch durch Lucas Cranach d. Ä. künstlerischem
      Talent und Produktivität. Martin Luther nahm sogar nachweislich die Dienste von 
      Lucas Cranach d. Ä. für Tiefenholzschnitte entgegen, welche sogar für
      theologische Argumentationen benutzt wurden. Lucas Cranach d. Ä. trug aber auch
      dazu bei, dass es ein öffentliches Bild von Martin Luther gab und versuchte dieses
      auch zu manifestieren. Als 1521 die Reichsacht über Martin Luther verhängt wurde, 
      musste er aus Schutz seinen Tod durch einen inszenierten Überfall vortäuschen.
      Sein enger und guter Freund Lucas Cranach d. Ä. war jedoch über die Inszenierung 
      informiert. 
      Als Martin Luther zurück nach Wittenburg kehrte, um die dortigen Unruhen zu besänftigen, 
      trat er unter dem Pseudonym \glqq Junker Jörg\grqq{} auf. 
      Auch hier nahm Lucas Cranach d. Ä. eine entscheidende Rolle ein, da er auch in dieser 
      Zeit Bildnisse von Martin Luther anfertigte. 
      Beispielsweise wurde zu dieser Zeit ein Bildnisholzschnitt
      von Martin Luther als Junker Jörg gefertigt, welche absichtlich zur Medienstrategie
      verwendet wurde. So überbrachte Lucas Cranach d. Ä. die Nachricht, dass Martin Luther
      den inszenierten Überfall überlebte und vermittelte damit auch das Bild eines 
      entschlossenen und visionären Mannes. [\cite{heydenreich2017lucas}]
      \newline
      In diesem Forschungsprojekt wird sich auf die Datenbasis von Martin Luther als
      Junker Jörg des Cranach Digital Archive bezogen. Zur Entwicklung einer Antwort auf
      die Forschungsfrage wird nicht die gesamte Datenbasis des Cranach Digital Archive
      benötigt, da eine geringe Datenmenge zur Entwicklung von Virtual Reality-Szenen
      bereits ausreichen. Das Team hinter \url{lucascranach.org} hat die Daten bereits
      digitalisiert und aufgearbeitet, weswegen Werke und Gemälde, die miteinander
      verwandt oder in Beziehung stehen, bereits miteinander verknüpft sind.
    \subsection{Konzept und Nutzung von Virtual Reality}
      
      % Grundlagen von VR (Was ist VR überhaupt) und warum ist es im Kontext dieser BA wichtig? 
      % - Darstellung der Kunstwerke in 3D
  \section{Theoretische Einordnung}
    \subsection{Aktueller Forschungsstand}
    \subsection{Virtual Reality im musealen Kontext}
      \begin{itemize}
        \item Aufgeworfene Fragen zu beantworten
        \item Anhand von Literaturstudie beantworten
      \end{itemize}
  \section{Prototyping}
    Das Prototyping ist eine Methode in der Softwareentwicklung, um schnell an
    Ergebnisse zu gelangen. Anhand eines Prototypen kann schnell erkannt werden,
    ob eine Idee technisch umsetzbar ist und was noch benötigt wird, um das Produkt
    zu verbessern. Je nach Fortschritt des Prototypen können auch schon Nutzererfahrungen
    und Feedback gesammelt werden.
    Diese Methode wird in diesem Forschungsprojekt angewandt, um mehrere Ergebnisse
    zu erzielen und zu erörtern, welche technischen Möglichkeiten heutzutage existieren.
    \subsection{Auswahl der Technologie}
    
    \subsection{Umsetzung der Softwareentwicklung}
  \section{Ergebnisse}
    \begin{itemize}
      \item Hier verwendet man beschriebene Methoden an
      \item Beschreiben, wie Untersuchung verlaufen ist
      \item Ergebnisse analysieren
    \end{itemize}
  \section{Diskussion und Fazit}
    \begin{itemize}
      \item Folgen und Ursache der Ergebnisse beschreiben
      \item Limitationen und Vorschläge für zukünftige Projekte darlegen
    \end{itemize}
    -
    \begin{itemize}
      \item Auf wichtigste Ergebnisse eingehen
      \item Geht auf Einleitung ein, da auf Forschungsfrage eingeht
      \item Füge keine neuen Informationen und Interpretationen
      \item Füge keine Beispiele und Zitate ein, bleibe bei den Fakten
      \item Dein Ergebnis ist immer wertvoll
      \item Ergebnisse deiner Forschung werden im Präsens geschrieben
    \end{itemize}
  \section{Anhang}
    \subsection{Abbildungen}
    \subsection{Literaturverzeichnis}
      \printbibliography
    \subsection{Abbildungsverzeichnis}
    \subsection{Eidesstattliche Erklärung} 
\end{document}